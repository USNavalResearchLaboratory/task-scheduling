\documentclass[11pt]{article}

\linespread{1.6}

%\usepackage[letterpaper,vmargin=1in,hmargin=1.25in]{geometry}
\usepackage[a4paper,width=150mm,top=25mm,bottom=25mm]{geometry}

\usepackage{microtype}

\usepackage{graphicx}
\graphicspath{{./Figures/}}

%\usepackage[backend=biber,style=numeric]{biblatex}
%\addbibresource{./References/references.bib}

\usepackage{amsmath,amssymb,amsfonts}
\usepackage{bm}
\usepackage{upgreek}
\usepackage[retainorgcmds]{IEEEtrantools}

\usepackage{hyperref}
\usepackage{cleveref}

\usepackage[colorinlistoftodos,backgroundcolor=yellow,linecolor=red]{todonotes}


%\usepackage{algorithm,algorithmicx}
%\usepackage{multirow}


\usepackage{etoolbox}
\undef{\Bbb}
\usepackage{mathfont_shortcuts}

\DeclareMathOperator*{\argmin}{arg\,min}
\DeclareMathOperator*{\argmax}{arg\,max}

\DeclareMathOperator{\Rbbgeq}{\mathbb{R}_{\geq 0}}
\DeclareMathOperator{\Zbbgeq}{\mathbb{Z}_{\geq 0}}


\title{Feature space design for supervised learning of optimal tasking schedules}
\author{Paul Rademacher}

\begin{document}

\maketitle
%\listoftodos


\section{Introduction}
Task scheduling is a mature field of research with a wide range of applications. Determining the optimal schedule given a set of tasks can be require a prohibitive amount of computation, especially when the scheduler is meant to operate in real time. 


\section{Problem Statement}
The scheduler operates on a set of $N$ tasks, $\{T_1, \ldots, T_N\}$; each task $T_i$ is defined by a release time $t_{\rrm, i}$, a duration $d_i$, and a monotonically non-decreasing loss function $l_i: [t_{\rrm,i}, \infty) \mapsto \Rbbgeq$. The objective is to jointly determine execution times $t_{\erm, i}$ for the tasks to minimize the overall loss. Together, the executions times $(t_{\erm, 1}, \ldots, t_{\erm, N})$ are referred to as a schedule.

\todo[inline]{assume zero loss at release time?}

The space of valid schedules is 
\begin{IEEEeqnarray}{rCl}
	\Tcal & = & \left\{\right\}
\end{IEEEeqnarray}


%\printbibliography

\end{document}





















